\documentclass[11pt]{article}

\title{\textbf{CASE STUDIES I: LibreSoft Convention}}
\author{Sergio Arroutbi Braojos}
\date{\today}
\usepackage{listings}
\addtolength{\voffset}{-50pt}
\begin{document}

\maketitle

\section{Introduction}
  2012 LibreSoft Convention took place last 29th November. In the first part of the convention, a quick introduction to the Libresoft group achievements on 2012 was introduced.\\
\\
  Libresoft, FLOSS research group at URJC in Madrid, collects and teaches knowledge about FLOSS, not only from the classic software development point of view, but also taking into consideration other aspects as FLOSS community, legal aspects, or an enterprise,industry and economic factors, in order to produce amazing results about FLOSS and its impact on the economy, technology and society.\\
  \\
  On year 2012, apart from continuing educating around FLOSS via MSWL Master, this group achieved some of its members, in particular Daniel Izquiero, reaching doctorate, it is also worth mentioning that some startups have been created around FLOSS. In particular, three of them:
\begin{itemize}
  \item{\textbf{Stack Sherpa}} : To provide different services, such as Cloud Services.
  \item{\textbf{Bitergia}} : Born to create the best reports around FLOSS statistics.
  \item{\textbf{FLOSSystems}} : To provide different services, such as Virtualization or High Availability.
\end{itemize}
After this short introduction, two guest speakers, presented below, shared their knowledge and experience with the rest of attendees. This two speakers were Andres Leonardo Martinez and Roberto Majadas.
  
\section{Andres Leonardo Martinez}
  Andres L. Martinez, former professor on Libresoft group, is currently a developer program manager.
  On the one hand, Andres shared some of his experiences running a developer program, where  expertise on open source community management is an essential skill.\\
\\
  On the other hand, he also explained about The Morfeo Project, and to which point a huge company, very separated from FLOSS, such as Telefonica, takes the decission of releasing some of the Software as FLOSS.\\
\\
  Last, but not least, Andres introduced the new Firefox OS, and how Mozilla is driving the project to attract the bigger community, the better.

\section{Roberto Majadas}
  Roberto Majadas is a long standing contributor in GNOME. Entrepreneur, developer and creator of a FLOSS startup named OpenShine, Roberto presented his experiences on his company.\\
\\
  On the one hand, Roberto stated how his view on the daily work was separated from the traditional software consultancy company he worked, and how creating your start-up can help a FLOSS developer to sort out this issue. Together with this aspect, Roberto clarified how different is learning to create a Software company from the software programming itself.\\
\\
  On the other hand, his presentation allowed attendees to explain OpenShine and how they have been collaborating with many big companies, such as Movistar, engaging them in FLOSS and being the protagonism of many FLOSS software releases.\\
\\
  Apart from that, an attendee initiated a debate on how difficult was starting-up companies on Spain, compared to other countries, and how difficult is launching a company in Spain. Roberto sorted the question comparing to the differences that supposed working as software developer in Spain compared to other countries, and how software programmers continue to develop Spain, for different reasons.
\end{document}
