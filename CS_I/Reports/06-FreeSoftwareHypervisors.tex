\documentclass[11pt]{article}

\title{\textbf{CASE STUDIES I: Free Software Hypervisors}}
\author{Sergio Arroutbi Braojos}
\date{\today}
\usepackage{listings}
\addtolength{\voffset}{-50pt}
\begin{document}

\maketitle

\section{Introduction}
Hypervisors are software pieces that allow multiple Operating System in a same machine, known as 'host' machine, to share hardware in paralel. Each of the Operating Systems are known as 'guest' systems.\\
\\
From the guest Operating System point, resources can be used as if it had access to them in a normal way, as if no hypervisor existed.\\
\\
Hypervisor mission is to control resources access in order to prevent guest Operating System to disrupt each others.\\
\\
Regarding type of existing hypervisors, basically two types exist:
\begin{itemize}
\item{\textbf{Bare Metal}}. Hypervisor runs on the bare hardware. Guest application executes system calls to Hypervisor, and Hypervisor directly operates on Hardware. \textbf{Best type in terms of performance}.
\item{\textbf{Hosted Hypervisor}}. Hypervisor runs on the Host Operating System. There is an extra layer compared to Bare Metal, which is the Operating System. This \textbf{additional layer means less performance}, as now, Hypervisor itself has to use system calls to the Host Operating System, which is the one that actions on hardware.
\end{itemize}

\section{Hypervisor Products}
\subsection{Xen}
XEN is a bare metal Hypervisor. This hypervisor is splitted in two "Domains", called DOM0 and DOMU:
\begin{itemize}
\item{DOM0}: Domain which contains a Kernel which is able to operate on Hardware, and is controllable by console. Acts as \textbf{control domain}.
\item{DOMU}: \textbf{Unprivileged Domain}. Contain guests Operating Systems.
\end{itemize}
Taking into account working modes for this hypervisor, Xen offers two possibilities:
\begin{enumerate}
\item{\textbf{Paravirtualization}}. In this mode, Guests Operating system know about the existance of Xen Hypervisor.\\
\textbf{Guest have to adapt their own kernels} and adapt system calls. It is considered a lightweight virtualization.\\ 
\textbf{No CPU virtualization extensions} are required from the host.\\
\textbf{Emulated hardware is not necessary}.\\
It is the \textbf{best type of virtualization in terms of performance}.
\item{\textbf{Full Virtualization}}. In this type of virtualization, Guest system must not be adapted.\\
\textbf{CPU hardware extensions required (VT/AMD-v)}\\
\textbf{Emulated hardware is needed}. Emulation is implemented via \textbf{Qemu}.\\
\textbf{It has a slower performance compared to paravirtualization}, but is easier to implement.
\end{enumerate}
Regarding license, Xen is licensed under \textbf{GPLv2}.
\subsection{KVM}
KVM (Kernel-based Virtual Machine) is a bare metal Hypervisor. This hypervisor only has a \textbf{fully virtualized model}, where each machine has private virtualized hardware.\\
\\
KVM is a set of Linux kernel modules, which provides both core virtualization infrastructure and a processor-specific module.\\
\textbf{Hardware emulation is done via Qemu}.\\
\\
KVM is licensed under various Free Software licenses:
\begin{itemize}
\item{KVM kernel module} : GPL v2
\item{KVM user module} : LGPL v2
\item{QEMU virtual library and QEMU PC system emulator} : LGPL
\item{Linux user mode QEMU emulator} : GPL
\item{BIOS files} : LGPL v2 or later
\end{itemize}
\subsection{VirtualBox}
VirtualBox is a Hosted Hypervisor. It is used for workstation purposes, being one of the most used products to test and evaluate Operating Systems different from the host one.\\
\\
VirtualBox uses Qemu in the same way as the already described hypervisors, mainly in two ways:
\begin{enumerate}
\item{To emulate some virtual devices}.
\item{Using the recompiler of Qemu as fallback mechanism}.
\end{enumerate}
\subsection{Qemu}
As stated before, Qemu is commonly used by hypervisors to emulate hardware. Qemu is not a Hypervisor itself, but a software that allows executing and booting both Operating Systems and programs which are for a certain machine (target) in other machine with different architecture (guest).\\
\\ 
Qemu has, basically, two working modes:
\begin{itemize}
\item{\textbf{Machine emulator}}: QEMU can boot and run Operating Systems made for one machine on a different machine, by means of dynamic translation.
\item{\textbf{Virtualizer}}: QEMU executes the guest code directly on the host CPU, in order to achieve native performance.
\end{itemize}
\end{document}
